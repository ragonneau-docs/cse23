%% cse23.tex
%% Copyright 2023 Tom M. Ragonneau
\documentclass[
    % nogiturl,  % Whether to include the git URL in the license frame
    % nolicense,  % Whether to include the license frame
]{presentation}
\usepackage{siunitx}

% Resources for the bibliography
\addbibresource{bib/strings.bib}
\addbibresource{bib/optim.bib}
\addbibresource{bib/ragonneau.bib}

% PGF/TikZ support
\usepackage{pgfplots}
\usepackage{pgfplotstable}
\usepackage{makecell}
\usetikzlibrary{calc,decorations.pathreplacing,overlay-beamer-styles,patterns,patterns.meta,shapes}
\pgfplotsset{
    compat=1.18,
}
\usepgfplotslibrary{external}
\pgfplotscreateplotcyclelist{profiles}{
    thick,mark=none,VeniceBlue,solid\\
    thick,mark=none,Caramel,dashed\\
    thick,mark=none,FernGreen,dotted\\
    thick,mark=none,Maroon,dashdotted\\
}
\pgfplotsset{
    every axis/.append style={
        label style={font=\footnotesize},
        legend cell align=left,
        legend style={
            rounded corners,
            thick,
            draw=black!15,
            font=\scriptsize,
        },
        tick label style={font=\footnotesize},
    },
}

% Quoting support
\usepackage[
    begintext=\openautoquote,
    endtext=\closeautoquote,
    vskip=\topsep,
]{quoting}

% Performance profiles
\usepackage{xstring}
\newcommand{\drawprofiles}[3]{%
    \def\selectsolvers{#1}%
    \def\selectcsv{figures/#2}%
    \def\selectprofile{#3}%
    \def\selectxlabel{$\log_2(\text{perf.\ ratio})$}%
    \def\selectylabel{Perf.\ profiles ($\tau = 10^{-#3}$)}%
    %% utils/profiles.tex
%% Copyright 2022 Tom M. Ragonneau
%
% This work may be distributed and/or modified under the
% conditions of the LaTeX Project Public License, either version 1.3
% of this license or (at your option) any later version.
% The latest version of this license is in
%   http://www.latex-project.org/lppl.txt
% and version 1.3 or later is part of all distributions of LaTeX
% version 2005/12/01 or later.
%
% This work has the LPPL maintenance status `maintained'.
%
% The Current Maintainer of this work is Tom M. Ragonneau.
\begin{tikzpicture}
    \pgfplotstableread[col sep=comma]{\selectcsv}\selectcsvread
    \newcommand{\getxmaxcsv}[1]{%
        \pgfplotstablegetrowsof{\selectcsvread}%
        \pgfmathtruncatemacro{\LastRowNo}{\pgfplotsretval-1}%
        \pgfplotstablesort[sort key={#1}]{\csvsorted}{\selectcsvread}%
        \pgfplotstablegetelem{\LastRowNo}{#1}\of{\csvsorted}%
        \pgfmathsetmacro{\selectxmaxcsv}{\pgfplotsretval}%
    }
    \pgfmathparse{\selectsolvers[0]}
    \let\selectsolver\pgfmathresult\relax
    \getxmaxcsv{x\selectprofile_\selectsolver}
    \begin{axis}[%
        width=185pt,%
        xmin=0,%
        xmax=0.55*\selectxmaxcsv,%
        ymin=0,%
        ymax=1,%
        minor y tick num=1,%
        yminorticks=true,%
        ytick={0,0.2,...,1},%
        cycle list name=profiles,%
        legend pos=south east,%
        xlabel={\selectxlabel},%
        ylabel={\selectylabel},%
        xticklabel style={/pgf/number format/1000 sep={}},%
        /pgfplots/max space between ticks=30,%
    ]
        \pgfmathparse{dim{\selectsolvers}-1}
        \let\selectlastindex\pgfmathresult\relax
        \foreach \i in {0,1,...,\selectlastindex} {%
            \pgfmathparse{\selectsolvers[\i]}%
            \let\selectsolver\pgfmathresult\relax%
            % \StrSubstitute{\selectsolver}{-}{~}[\selectsolverescaped]%
            \addplot table[%
                x=x\selectprofile_\selectsolver,%
                y=y\selectprofile,%
                col sep=comma,%
            ]{\selectcsvread};%
            % \addlegendentryexpanded{\selectsolverescaped}%
            \addlegendentryexpanded{\selectsolver}%
        }
    \end{axis}
\end{tikzpicture}%
}

% Maths commands for this presentation
\newcommand{\obj}{f}
\newcommand{\objm}[1][]{\hat{f}\ifblank{#1}{}{^{#1}}}
\newcommand{\con}{c}
\newcommand{\conm}[1][]{\hat{c}\ifblank{#1}{}{^{#1}}}
\newcommand{\iter}[1][]{x\ifblank{#1}{}{^{#1}}}
\newcommand{\lag}{\mathcal{L}}
\newcommand{\lagm}[1][]{\hat{\mathcal{L}}\ifblank{#1}{}{^{#1}}}
\newcommand{\lm}[1][]{\lambda\ifblank{#1}{}{^{#1}}}
\newcommand{\nstep}[1][]{n\ifblank{#1}{}{^{#1}}}
\newcommand{\rad}[1][]{\Delta\ifblank{#1}{}{^{#1}}}
\newcommand{\step}[1][]{d\ifblank{#1}{}{^{#1}}}
\newcommand{\tstep}[1][]{t\ifblank{#1}{}{^{#1}}}
\newcommand{\xl}{l}
\newcommand{\xpt}[1][]{\mathcal{Y}\ifblank{#1}{}{^{#1}}}
\newcommand{\xu}{u}

% Metadata
\title{COBYQA}
\subtitle{A derivative-free trust-region SQP method for nonlinearly constrained optimization}
\date{CSE23 (March 2, 2023)}
\author{\href{https://www.tomragonneau.com/}{\textbf{Tom M. Ragonneau}} \and \href{https://www.zhangzk.net/}{Zaikun Zhang}}
\institute{
    Department of Applied Mathematics\\
    The Hong Kong Polytechnic University
}
\titlegraphic{}
\hypersetup{
    pdfsubject={Presentation of COBYQA at CSE23},
    pdfkeywords={COBYQA,optimization,constrained-optimization,nonlinear-optimization,derivative-free optimization,blackbox optimization,trust-region method,SQP framework},
}

\begin{document}

\maketitle

\begin{frame}{General context}
    We design a method named COBYQA for solving\footnote{The equality constraints are omitted here for simplicity.}
    \begin{align*}
        \min_{\iter \in \R^n}   & \quad \obj(\iter)\\
        \mathrm{s.t.}           & \quad \con(\iter) \le 0,\\
                                & \quad \xl \le \iter \le \xu,
    \end{align*}
    where derivatives of $\obj$ and $\con$ are \alert{unavailable}.

    \medskip

    \begin{block}{Notes on the method}
        \begin{itemize}
            \item COBYQA aims at being a \alert{successor} to COBYLA \parencite{Powell_1994}.
            \item We \alert{implement} COBYQA into a \href{https://pypi.org/project/cobyqa/}{Python solver}.
            \item The bound constraints are handled \alert{separately}.
        \end{itemize}
    \end{block}
\end{frame}

\begin{frame}{Inviolable bound constraints}
    \begin{block}{The bound constraints $\xl \le \iter \le \xu$ are assumed inviolable}
        \begin{itemize}
            \item They often represent \alert{inalienable} restrictions.
            \item $\obj$ or $\con$ may not be defined otherwise.
        \end{itemize}
    \end{block}

    \medskip

    Therefore, COBYQA \alert{always} respects the bound constraints.

    \medskip

    \begin{block}{A few examples from academia and industry}
        \begin{itemize}
            \item Optimization method tuning \parencite{Audet_Orban_2006}.
            \item Hyperparameter tuning \parencite{Ghanbari_Scheinberg_2017}.
            \item Aircraft engineering \parencite{Gazaix_Etal_2019}.
        \end{itemize}
    \end{block}
\end{frame}

\begin{frame}{Table of contents}
    \tableofcontents[hideallsubsections]
\end{frame}

\section{The general framework}

\begin{frame}{The derivative-free trust-region SQP method}

    COBYQA iteratively solves the trust-region SQP subproblem
    \begin{align*}
        \min_{\step \in \R^n}   & \quad \nabla \only<1>{\obj}\only<2>{\alert{\objm[k]}}(\iter[k])^{\T} \step + \frac{1}{2} \step^{\T} \nabla_{x, x}^2 \only<1>{\lag}\only<2>{\alert{\lagm[k]}}(\iter[k], \lm[k]) \step\\
        \textrm{s.t.}           & \quad \only<1>{\con}\only<2>{\alert{\conm[k]}}(\iter[k]) + \nabla \only<1>{\con}\only<2>{\alert{\conm[k]}}(\iter[k]) \step \le 0,\\
                                & \quad \xl \le \iter[k] + \step \le \xu,\\
                                & \quad \norm{\step} \le \rad[k],
    \end{align*}
    with $\only<1>{\lag}\only<2>{\alert{\lagm[k]}}(\iter, \lm) = \only<1>{\obj}\only<2>{\alert{\objm[k]}}(\iter) + \lm^{\T} \only<1>{\con}\only<2>{\alert{\conm[k]}}(\iter)$\only<2>{, given some models $\objm[k]$ and $\conm[k]$}.

    \medskip
    \pause

    \begin{block}{Remarks on this subproblem}
        \begin{itemize}
            \item We only require an approximate solution $\step[k]$.
            \item The solution must satisfy $\xl \le \iter[k] + \step[k] \le \xu$.
            \item The subproblem may be \alert{infeasible}. What is a solution?
        \end{itemize}
    \end{block}
\end{frame}

\begin{frame}{A new \citeauthor{Byrd_1987}-\citeauthor{Omojokun_1989} approach}
    We compute $\step[k] = \nstep[k] + \tstep[k]$, where
    \begin{itemize}
        \item the \alert{normal} step $\nstep[k]$ reduces the (possible) constraint violation, and
        \item the \alert{tangential} step $\tstep[k]$ reduces the quadratic objective function.
    \end{itemize}

    \begin{columns}
        \begin{column}{0.4\textwidth}
            \begin{center}
                \begin{tikzpicture}[scale=0.75]
                    % Linear constraints
                    \uncover<1,2,6->{\fill[color=MidnightBlue,opacity=0.4] (-4,-1) -- (-1.5,-1) -- (-0.5,4) -- (-4,4) -- cycle;}
                    \uncover<3-5>{\fill[color=MidnightBlue,opacity=0.4] (-4,-1) -- (-2.1,-1) -- (-1.1,4) -- (-4,4) -- cycle;}
                    \uncover<1,2,6>{\fill[color=MidnightBlue,opacity=0.4] (-4,1) -- (0,4) -- (-4,4) -- cycle;}
                    \uncover<3-5,7->{\fill[color=MidnightBlue,opacity=0.4] (-4,0.125) -- (1,3.875) -- (1,4) -- (-4,4) -- cycle;}

                    % Trust regions
                    \begin{scope}
                        \clip (-4,-1) rectangle (1,4);
                        \draw[fill=Caramel!80,draw opacity=0.7,fill opacity=0.5] (0,0) circle (3);
                        \draw[densely dotted,fill=Caramel!80,opacity=0.7] (0,0) circle (2.5);
                    \end{scope}

                    % Feasible region for the tangential subproblem
                    \begin{scope}
                        \clip (-4,0.125) -- (-1.5,2) -- (-1.1,4) -- (-4,4) -- cycle;
                        \uncover<4-5>{\fill[pattern=north west lines,pattern color=VeniceBlue!70!black,opacity=0.8] (0,0) circle (3);}
                    \end{scope}
                    \begin{scope}
                        \clip (-4,0.125) -- (-27/34,43/17) -- (-0.5,4) -- (-4,4) -- cycle;
                        \uncover<8->{\fill[pattern=north west lines,pattern color=VeniceBlue!70!black,opacity=0.8] (0,0) circle (3);}
                    \end{scope}

                    % Frame and annotations
                    \uncover<5>{
                        \draw[-stealth,thick,FernGreen] (-1.5,2) -- (-2.2,1.7);
                        \node[below,xshift=9.5pt,text=FernGreen] at (-1.85,1.85) {$\tstep[k]$};
                        \draw[-stealth,thick] (0,0) -- (-2.2,1.7);
                        \node[below left] at (-1.1,0.85) {$\step[k]$};
                    }
                    \uncover<9>{
                        \draw[-stealth,thick,FernGreen] (-1.5,2) -- (-0.9,2.7);
                        \node[below,xshift=5pt,text=FernGreen] at (-1.2,2.35) {$\tstep[k]$};
                        \draw[-stealth,thick] (0,0) -- (-0.9,2.7);
                        \node[above right] at (-0.45,1.35) {$\step[k]$};
                    }
                    \uncover<2->{\draw[-stealth,thick,Maroon] (0,0) -- (-1.5,2);}
                    \uncover<2-5>{\node[above right,text=Maroon] at (-0.75,1) {$\nstep[k]$};}
                    \uncover<6->{\node[below left,text=Maroon] at (-0.75,1) {$\nstep[k]$};}
                    \draw[fill] (0,0) circle (1.4pt) node[below right] {$\iter[k]$};
                    \draw[thick] (-4,-1) rectangle (1,4);
                \end{tikzpicture}
            \end{center}
        \end{column}
        \begin{column}{0.6\textwidth}
            \begin{tikzpicture}
                \draw[fill=Caramel!80,draw opacity=0.7,fill opacity=0.5] (0,0) circle (.2);
                \node[right] at (.4,0) {Trust region};
                \draw[densely dotted,fill=Caramel!80,opacity=0.7] (0,-0.6) circle (.2);
                \node[right] at (.4,-0.6) {Reduced trust region};
                \fill[color=MidnightBlue,opacity=0.4] (-.2,-1.4) rectangle (.2,-1);
                \node[right] at (.4,-1.2) {Linear constraints};
                \uncover<4->{
                    \fill[pattern=north west lines,pattern color=VeniceBlue!70!black,opacity=0.8] (-.2,-2) rectangle (.2,-1.6);
                    \node[right] at (.4,-1.8) {Feasible region for $\tstep[k]$};
                }
            \end{tikzpicture}

            \bigskip

            The \alert<1-5>{standard} approach\footnotemark\ vs.\ the \alert<6->{new} one.
        \end{column}
    \end{columns}

    \uncover<9>{The feasible region for $\tstep[k]$ is \alert{wider} in the new approach.}
    \footnotetext{\textcite[\S 15.4.4]{Conn_Gould_Toint_2000}.}
\end{frame}

\begin{frame}{A new \citeauthor{Byrd_1987}-\citeauthor{Omojokun_1989} approach (cont'd)}
    The \alert{\only<1>{standard}\only<2>{new}} approach is as follows.
    \begin{itemize}
        \item The normal step $\nstep[k]$ solves approximately (for some $\zeta < 1$)
        \begin{align*}
            \min_{\step \in \R^n}   & \quad \norm[\big]{\big[ \conm[k](\iter[k]) + \nabla \conm[k](\iter[k]) \step \big]^+}\\
            \textrm{s.t.}           & \quad \xl \le \iter[k] + \step \le \xu,\\
                                    & \quad \norm{\step} \le \zeta \rad[k].
        \end{align*}
        \item The tangential step $\tstep[k]$ solves approximately
        \begin{align*}
            \min_{\step \in \R^n}   & \quad \big[ \nabla \objm[k](\iter[k]) + \nabla_{x, x}^2 \lagm[k](\iter[k], \lm[k]) \nstep[k] \big]^{\T} \step + \frac{1}{2} \step^{\T} \nabla_{x, x}^2 \lagm[k](\iter[k], \lm[k]) \step\\
            \textrm{s.t.}           & \quad \nabla \conm[k](\iter[k])^{\T} \step \le \alert{\only<1>{0}\only<2>{[\conm[k](\iter[k]) + \nabla \conm[k](\iter[k]) \nstep[k]]^-}},\\
                                    & \quad \xl \le \iter[k] + \nstep[k] + \step \le \xu,\\
                                    & \quad \norm{\nstep[k] + \step} \le \rad[k].
        \end{align*}
    \end{itemize}
\end{frame}

\section{Interpolation-based models}

\begin{frame}{Interpolation-based quadratic models}
    COBYQA models $\obj$ and $\con$ by \alert{quadratic} interpolation, as follows.\footnote{Other methods: \textcite{Conn_Scheinberg_Toint_1997a,Conn_Scheinberg_Toint_1997b,Conn_Scheinberg_Toint_1998}, \textcite{Wild_2008}, \textcite{Bandeira_Scheinberg_Vicente_2012}, \textcite{Zhang_2014}, \textcite{Xie_Yuan_2023}.}

    \begin{block}{Derivative-free symmetric Broyden update \parencite{Powell_2004b}}
        The $k$th model $\objm[k]$ of $\obj$ solves
        \begin{align*}
            \min_Q          & \quad \norm[\big]{\nabla^2 \objm[k - 1] - \nabla^2 Q}_{\mathsf{F}}\\
            \textrm{s.t.}   & \quad Q(y) = \obj(y), ~ y \in \xpt[k],
        \end{align*}
        for some $\xpt[k] \subseteq \R^n$, with $\objm[-1] \equiv 0$.
        The model $\conm[k]$ of $\con$ is built similarly.
    \end{block}

    \vspace{-0.25\baselineskip}
    The interpolation set $\xpt[k]$ is \alert{recycled} at each iteration.
    \begin{itemize}
        \item The set $\xpt[k + 1]$ is built as $(\xpt[k] \setminus {\bar{y}}) \cup \set{\iter[k] + \step[k]}$ for some $\bar{y} \in \xpt[k]$.
        \item The KKT system of this variational problem is \alert{linear}.
    \end{itemize}
\end{frame}

\section{Many difficulties arise}

\begin{frame}{A lot of questions must be addressed}
    \begin{itemize}
        \item How to calculate the steps $\nstep[k]$ and $\tstep[k]$ numerically?\\
        \textcolor{FernGreen}{COBYQA adapts the truncated conjugate gradient method.}
        \item What is the approximate Lagrange multiplier $\lm[k]$?\\
        \textcolor{FernGreen}{We choose the least-squares Lagrange multiplier.}
        \item Which merit function should we use?\\
        \textcolor{FernGreen}{COBYQA uses the $\ell_2$-merit function.}
        \item How to update the penalty parameter?\\
        \textcolor{FernGreen}{The update incorporates}
        \begin{itemize}
            \item \textcolor{FernGreen}{a theoretical value for the exactness of the merit function, and}
            \item \textcolor{FernGreen}{a strategy used by Powell in COBYLA.}
        \end{itemize}
    \end{itemize}

    \bigskip

    These questions (and more) are addressed in \textcite{Ragonneau_2022}.
\end{frame}

\begin{frame}{A crucial difficulty in the implementation}
    \begin{itemize}
        \item What if the interpolation set $\xpt[k]$ is almost nonpoised?\\
        \textcolor{FernGreen}{A well-known approach: using a geometry-improving mechanism.\footnote{\textcite{Conn_Scheinberg_Vicente_2008a,Conn_Scheinberg_Vicente_2008b}, \textcite{Fasano_Morales_Nocedal_2009}.}}
    \end{itemize}

    \medskip

    \begin{block}{This is a central difficulty in the implementation of DFO methods}
        \begin{center}
            \begin{tikzpicture}
                \draw[thick] (0,0) rectangle (3,1.5);
                \draw[thick] (6,0) rectangle (9,1.5);
                \draw[thick,Maroon] (3,1) -- (4.4,1);
                \draw[-stealth,thick,Maroon] (4.6,1) -- (6,1);
                \draw[thick,Maroon] (4.3,0.9) -- (4.5,1.1);
                \draw[thick,Maroon] (4.5,0.9) -- (4.7,1.1);
                \draw[thick,Maroon] (6,0.5) -- (4.6,0.5);
                \draw[-stealth,thick,Maroon] (4.4,0.5) -- (3,0.5);
                \draw[thick,Maroon] (4.3,0.4) -- (4.5,0.6);
                \draw[thick,Maroon] (4.5,0.4) -- (4.7,0.6);
                \node at (1.5,0.75) {\makecell{Modeling\\ process}};
                \node at (7.5,0.75) {\makecell{Optimization\\ process}};
                \node[above,text=Maroon] at (4.5,1.1) {\small\emph{Often inhibit}};
                \node[below,text=Maroon] at (4.5,0.4) {\small\emph{each other}};
            \end{tikzpicture}
        \end{center}

        \begin{itemize}
            \item The iterates~$\set{x^k}$ likely lie on a particular path.
            \item The modeling process does \alert{not} ponder the optimization problem.
        \end{itemize}
    \end{block}
\end{frame}

\section{Implementation and experiments}

\begin{frame}{The Python implementation of COBYQA}
    \begin{block}{A quote from \textcite{Powell_2006}}
        \begin{quoting}
            \small%
            The development of NEWUOA has taken nearly \alert{three years}.
            The work was very \alert{frustrating} [...]
        \end{quoting}
        The development of COBYQA was not easier.
    \end{block}

    \smallskip

    We implemented COBYQA in \alert{Python} and made it publicly available.

    \smallskip

    \begin{columns}
        \begin{column}{0.5\textwidth}
            \begin{center}
                \qrcode{https://www.cobyqa.com/}\\[1ex]
                \href{https://www.cobyqa.com/}{Documentation}
            \end{center}
        \end{column}
        \begin{column}{0.5\textwidth}
            \begin{center}
                \qrcode{https://github.com/cobyqa/cobyqa/}\\[1ex]
                \href{https://github.com/cobyqa/cobyqa/}{Source code}
            \end{center}
        \end{column}
    \end{columns}
\end{frame}

\begin{frame}{Comparing COBYQA with existing DFO solvers}
    \begin{itemize}
        \item We assess the quality of points based on
        \begin{empheq}[left={\varphi(\iter) = \empheqlbrace}]{alignat*=2}
            & \obj(\iter)                           && \quad \text{if~$v_{\infty}(x) \le 10^{-10}$,}\\
            & \infty                                && \quad \text{if~$v_{\infty}(x) \ge 10^{-5}$,}\\
            & \obj(\iter) + 10^5 v_{\infty}(\iter)  && \quad \text{otherwise,}
        \end{empheq}
        where~$v_{\infty}$ denotes the~$\ell_{\infty}$-constraint violation.
        \item The problems are from the \alert{CUTEst} set.
        \item The problems are of \alert{dimension} at most \num{50} (this is \alert{not} small).
        \item The noisy problems replace~$\obj$ with
        \begin{equation*}
            \tilde{\obj}(x) = \big[1 + \epsilon(\iter)\big] \obj(\iter),
        \end{equation*}
        where~$\epsilon(x) \sim \mathcal{N}(0, \sigma^2)$.
    \end{itemize}
\end{frame}

\begin{frame}{Performance of the new \citeauthor{Byrd_1987}-\citeauthor{Omojokun_1989} approach}
    We compare the new and the standard \citeauthor{Byrd_1987}-\citeauthor{Omojokun_1989} approaches
    \begin{itemize}
        \item on \alert{linearly} and \alert{nonlinearly constrained} problems,
        \item in the implementation of COBYQA.
    \end{itemize}

    \smallskip

    \begin{center}
        \drawprofiles{{"New","Standard"}}{plain-1-50-perf-byrd-omojokun-nlqo.csv}{4}
    \end{center}
\end{frame}

\begin{frame}{Performance on bound-constrained problems}
    We compare COBYQA, COBYLA, and two implementations of BOBYQA
    \begin{itemize}[<+->]
        \item on \alert{bound-constrained} problems,
        \item adding \alert{noise} to~$\obj$, with~$\sigma = 10^{-3}$.
    \end{itemize}

    \smallskip

    \begin{center}
        \only<1>{\drawprofiles{{"COBYQA","BOBYQA","COBYLA","Py-BOBYQA"}}{plain-1-50-perf-bobyqa-cobyla-cobyqa-py-bobyqa-b.csv}{4}}%
        \only<2>{\drawprofiles{{"COBYQA","BOBYQA","COBYLA","Py-BOBYQA"}}{noisy-1-50-3-perf-bobyqa-cobyla-cobyqa-py-bobyqa-b.csv}{2}}
    \end{center}
\end{frame}

\begin{frame}{Performance on nonlinearly constrained problems}
    We compare COBYQA and COBYLA
    \begin{itemize}[<+->]
        \item on \alert{nonlinearly constrained} problems,
        \item adding \alert{noise} to~$\obj$, with~$\sigma = 10^{-3}$.
    \end{itemize}

    \smallskip

    \begin{center}
        \only<1>{\drawprofiles{{"COBYQA","COBYLA"}}{plain-1-50-perf-cobyla-cobyqa-qo.csv}{4}}%
        \only<2>{\drawprofiles{{"COBYQA","COBYLA"}}{noisy-1-50-3-perf-cobyla-cobyqa-qo.csv}{2}}
    \end{center}
\end{frame}

\begin{frame}{Comparison with COBYLA}
    We compare COBYQA and COBYLA on \alert{all} problems.

    \bigskip

    \begin{center}
        \drawprofiles{{"COBYQA","COBYLA"}}{plain-1-50-perf-cobyla-cobyqa-ubnlqo.csv}{4}
    \end{center}
\end{frame}

\section{Conclusion}

\begin{frame}{Conclusion}

    \begin{itemize}
        \item COBYQA already received \alert{positive} feedback from practitioners.
        \item It will soon be included in
        \begin{itemize}
            \item \href{https://www.pdfo.net/}{PDFO} as a successor for COBYLA, and
            \item \href{https://gemseo.readthedocs.io/}{GEMSEO}, an \alert{industrial} software package for MDO.
        \end{itemize}
        \item We will soon investigate the convergence properties of COBYQA.
    \end{itemize}

    For more information, visit:

    \medskip

    \begin{columns}
        \begin{column}{\textwidth/3}
            \begin{center}
                \qrcode{https://www.cobyqa.com/}\\[1ex]
                \href{https://www.cobyqa.com/}{COBYQA's website}
            \end{center}
        \end{column}
        \begin{column}{\textwidth/3}
            \begin{center}
                \qrcode{https://www.tomragonneau.com/}\\[1ex]
                \href{https://www.tomragonneau.com/}{My website}
            \end{center}
        \end{column}
        \begin{column}{\textwidth/3}
            \begin{center}
                \qrcode{https://tomragonneau.com/documents/thesis.pdf}\\[1ex]
                \href{https://tomragonneau.com/documents/thesis.pdf}{My thesis}
            \end{center}
        \end{column}
    \end{columns}
\end{frame}

\appendix

\begin{frame}[t,allowframebreaks]{References}
    \printbibliography[heading=none]
\end{frame}

\end{document}
